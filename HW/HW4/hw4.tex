\documentclass{article}
\usepackage{xeCJK,amsmath,geometry,graphicx,amssymb,zhnumber,booktabs,setspace,tasks,verbatim,amsthm,amsfonts,mathdots}
\usepackage{listings,xcolor,float,caption,subfigure}
\geometry{a4paper,scale=0.8}   
\renewcommand\arraystretch{2}
\title{ICS  HomeWork-4}
\author{PB20000113孔浩宇}
\begin{document}
\maketitle
\section*{T1}
\begin{enumerate}
    \item [(1)]ADD
    \begin{enumerate}
        \item []指令类型:运算
        \item []寻址模式:寄存器寻址 (操作数1)
        \item []寻址模式:寄存器寻址或立即数寻址 (操作数2)
    \end{enumerate}
    \item [(2)]JMP
    \begin{enumerate}
        \item []指令类型:控制
        \item []寻址模式:寄存器寻址
    \end{enumerate}
    \item [(3)]LEA
    \begin{enumerate}
        \item []指令类型:数据搬移
        \item []寻址模式:立即数寻址
    \end{enumerate}
    \item [(4)]LDR
    \begin{enumerate}
        \item []指令类型:数据搬移
        \item []寻址模式:基址偏移寻址
    \end{enumerate}
    \item [(5)]NOT
    \begin{enumerate}
        \item []指令类型:运算
        \item []寻址模式:寄存器寻址
    \end{enumerate}
\end{enumerate}

\section*{T2}
    \begin{enumerate}
        \item [a.]至少需要8bits 宽度来表达地址
        \item [b.]至少要留出6位来标识PC相对偏移
        \item [c.]6.
    \end{enumerate}

\section*{T3}
\begin{enumerate}
    \item [a.]$0101\ 011\ 010\ 1\ 00100$. (AND\ $R_3$,\ $R_2$,\ \# 4)
    \item [b.]$0101\ 011\ 010\ 1\ 01100$. (AND\ $R_3$,\ $R_2$,\ \# 12)
    \item [c.]$1001\ 011\ 010\ \ 111111$. (NOT\ $R_3$,\ $R_2$)
    \item [d.]不能,因为AND中立即数仅有5位,如果要实现判断机器6是否为busy需要立即数为$100\ 0000$,无法实现。
\end{enumerate}

\section*{T4}
    \begin{enumerate}
        \item [a.]指令a不可以,在执行节拍执行了加法,虽然是加0.
        \item [b.]指令b不可以,会导致PC进行跳转,影响程序的正常执行。
        \item [c.]指令c可以,当nzp均0时,BR指令执行节拍什么都不做,也不影响程序的执行。
    \end{enumerate}
    
\section*{T5}
\begin{enumerate}
    \item [a.]$0001\ 011\ 010\ 1\ 00000\ $ (ADD\ $R_3,\ R_2,\ \# 0$)
    \item [b.]
    \begin{align*}
        &1001\ 011\ 011\ 111111\  (\mbox{NOT}\ R_3,\ R_3)\\
        &0001\ 011\ 011\ 1\ 00001\ (\mbox{ADD}\ R_3,\ R_3,\ \# 1)\\
        &0001\ 001\ 010\ 0\ 00011\ (\mbox{ADD}\ R_1,\ R_2,\ R_3)
    \end{align*}
    \item [c.]$0001\ 001\ 001\ 1\ 00000\ (\mbox{ADD}\ R_1,\ R_1,\ \# 0)$
    \item [d.]不能,无法使寄存器同时为负和零
    \item [e.]$0101\ 010\ 010\ 1\ 00000$\ (AND\ $R_2$,\ $R_2$,\ \# 0)
\end{enumerate}

\section*{T6}
\begin{enumerate}
    \item [(1):]$1001\ 100\ 001\ 111 111$
    \item [(2):]$0101\ 100\ 100\ 0\ 00\ 010$
    \item [(3):]$1001\ 101\ 010\ 111 111$
    \item [(4):]$0001\ 101\ 101\ 0\ 00\ 001$
    \item [(5):]$1101\ 011\ 100\ 0\ 00\ 101$
\end{enumerate}

\section*{T7}
\begin{enumerate}
    \item [$R_1$:]0x 3121
    \item [$R_2$:]0x 4566
    \item [$R_3$:]0xabcd
    \item [$R_4$:]0xabcd
\end{enumerate}

\section*{T8}
    \begin{enumerate}
        \item [a.]LD:2次;不需要执行节拍
        \item [b.]LDI:3次;不需要执行节拍
        \item [c.]LEA:1次;不需要取执行节拍
    \end{enumerate}

\section*{T9}
    $R_2$:0x 1482

\section*{T10}
    $R_0$的低8位里有5位为1.

\section*{T11}
    MUL 最有必要,因为其他指令都可以用已有的指令实现。
    \begin{enumerate}
        \item [a.]MOVE:可以用ADD立即数0或者AND来实现。
        \item [b.]NAND:可以用ADD与NOT连续两条指令来实现。
        \item [c.]SHFL:可以用 MUL$\ R_i,\ R_j,\ R_k (R_k=100)$。
    \end{enumerate}

\section*{T12}

\end{document}
 