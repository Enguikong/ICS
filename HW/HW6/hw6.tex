\documentclass{article}
\usepackage{xeCJK,amsmath,geometry,graphicx,amssymb,zhnumber,booktabs,setspace,tasks,verbatim,amsthm,amsfonts,mathdots}
\usepackage{listings,xcolor,float,caption,subfigure}
\geometry{a4paper,scale=0.8}   
\renewcommand\arraystretch{2}
\title{ICS  HomeWork-6}
\author{PB20000113孔浩宇}
\begin{document}
\maketitle
\section*{T1}
\begin{enumerate}
    \item [(a)]
    \begin{lstlisting}[basicstyle=\ttfamily,language={[x86masm]Assembler}]
            SaveRegisters   ST  R3, SAVER3 
                            ST  R4, SAVER4
                            ST  R5, SAVER5 
                            ST  R6, SAVER6
                            RET 
        RestoreRegisters    LD  R3, SAVER3 
                            LD  R4, SAVER4
                            LD  R5, SAVER5 
                            LD  R6, SAVER6 
                            RET
        SAVER3 .BLKW 1
        SAVER4 .BLKW 1
        SAVER5 .BLKW 1
        SAVER6 .BLKW 1
    \end{lstlisting}
    \item [(b)]
\end{enumerate}

\section*{T2}
\begin{enumerate}
    \item [(a)]
    \begin{lstlisting}[basicstyle=\ttfamily,language={[x86masm]Assembler}]
        CLEAR   ST  R2, TEMP
                LEA R2, MASKS
                ADD R2, R1, R2
                LDR R2, R2, #0
                NOT R2, R2
                AND R0, R2, R0
                LD  R2, TEMP
                RET
        TEMP    .BLKW #1
    \end{lstlisting}

    \item [(b)]
    \begin{lstlisting}[basicstyle=\ttfamily,language={[x86masm]Assembler}]
        SET     ST  R2, TEMP
                LEA R2, MASKS
                ADD R2, R1, R2
                LDR R2, R2, #0
                NOT R2, R2
                NOT R0, R0
                AND R0, R2, R0
                NOT R0, R0
                LD  R2, TEMP
                RET
	    TEMP    .BLKW #1
    \end{lstlisting}
\end{enumerate}

\section*{T3}
\begin{enumerate}
    \item [(a)]16.
    \item [(b)]x400F.
    \item [(c)]程序执行后,C中保存的值为从B中保存的内存位置开始的四个连续值的平均值。
\end{enumerate}



\section*{T4}
\begin{lstlisting}[basicstyle=\ttfamily,language={[x86masm]Assembler}]
    FACT	ST  R1, SAVE_R1
            ADD R1, R0, #0
            BRnp SKIP
            ADD R1, R1, #1
            BRnzp DONE
    SKIP	ADD R0, R0, #-1
            BRz DONE
    AGAIN	MUL R1, R1, R0
            ADD R0, R0, #-1     ; R0 gets next integer for MUL
            BRnp AGAIN
    DONE	ADD R0, R1, #0      ; Move n! to R0
            LD  R1, SAVER1
            RET
    SAVE_R1	.BLKW 1
\end{lstlisting}


\section*{T5}
\begin{enumerate}
    \item [(a)]若KBSR[15]为0,则在KBDR中的数据为已读的,不检查KBSR的话程序会再次读入同一个字符
    \item [(b)]若KBSR[15]为1,则用户此前输入的存储在KBDR中的未读数据会丢失
    \item [(c)]第二个更可能发生,因为程序读取一般不会连续读取,键盘输入可能连续输入
\end{enumerate}


\section*{T6}
对于合并的状态状态寄存器,检查指定的状态位是否为1需要更多的指令,我认为是个不好的设计。


\section*{T7}
\begin{enumerate}
    \item [(a)]TRAP向量是8位宽。LC-3可以实现256个TRAP例程
    \item [(b)]若IR中已存入TRAP,则只有1次;否则2次
\end{enumerate}


\section*{T8}
输出为FUN。


\section*{T9}
如果A中的值是质数,则内存位置RESULT中存入1;否则,RESULT中存入0。


\section*{T10}
\begin{enumerate}
    \item [(a)]
    \begin{lstlisting}[basicstyle=\ttfamily,language={[x86masm]Assembler}]
        ADD R1, R1, #1
    \end{lstlisting}

    \item [(b)]
    \begin{lstlisting}[basicstyle=\ttfamily,language={[x86masm]Assembler}]
        TRAP x25
    \end{lstlisting}

    \item [(c)]
    \begin{lstlisting}[basicstyle=\ttfamily,language={[x86masm]Assembler}]
        ADD R0, R0, #5
    \end{lstlisting}

    \item [(d)]
    \begin{lstlisting}[basicstyle=\ttfamily,language={[x86masm]Assembler}]
        BRzp K
    \end{lstlisting}
\end{enumerate}

\section*{T11}
\begin{enumerate}
    \item [(a)]键盘中断启用,数字2重复打印。
    \item [(b)]在屏幕上将键盘输入的字符显示两次。
    \item [(c)]先是打印2,然后键盘输入的字符显示2次或3次,之后连续打印2。
    \item [(d)]如果程序在指令LD\ R0,\ B执行后中断,则显示3次;否则2次。
\end{enumerate}


\section*{T12}
\begin{enumerate}
    \item [(1)]privilege mode 和状态码没有恢复。
    \item [(2)]由于R7中的值用于保存子程序结束后的返回地址,而不是保存在堆栈上的,因此程序很可能无法回到正确的位置。
\end{enumerate}

\end{document}

\begin{lstlisting}[basicstyle=\ttfamily,language={[x86masm]Assembler}]

\end{lstlisting}
