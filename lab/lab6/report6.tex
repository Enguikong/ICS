\documentclass[UTF8]{ctexart}

\usepackage{amsmath}
\usepackage{cases}
\usepackage{cite}
\usepackage{graphicx}
\usepackage[margin=1in]{geometry}
\usepackage{float}
\geometry{a4paper}
\usepackage{fancyhdr}
\usepackage{multirow}
\usepackage{listings,xcolor,caption,subfigure}
\pagestyle{fancy}
\fancyhf{}


\title{ICS-lab6实验报告}
\author{孔浩宇 PB20000113}
\date{\today}
\pagenumbering{arabic}

\begin{document}

\fancyhead[L]{孔浩宇}
\fancyhead[C]{ICS-lab6实验报告}
\fancyfoot[C]{\thepage}

\maketitle
\tableofcontents
\newpage

\section{实验目的}
    使用高级编程语言实现前四次实验的内容:
    \begin{enumerate}
        \item [1.]lab1: counting how many 1
        \item [2.]lab2: a variant of the fibonacci sequence
        \item [3.]lab3: longest duplicate substring
        \item [4.]lab4: sort and count
    \end{enumerate}


\section{实验原理}
    \subsection{判断A[15:0]第$i$位是否为1}
        取$E_i = 2^{i} $,则
            \[
                E_i \& A[15:0] = E_i \cdot A[i]
                \Rightarrow
                \begin{cases}
                    \ E_i \& A[15:0] == 1 &\Leftrightarrow\ A[i] == 1 \\
                    \ E_i \& A[15:0] == 0 &\Leftrightarrow\ A[i] == 0 
                \end{cases}
            \]

    \subsection{对$F[15:0]$取模$p$}
        \subsubsection{一般情况 ($2\leq p < 2^{16}$)}
            执行以下算法,取$f=F[15:0]$
            \begin{enumerate}
                \item [(1)]$f=f-p$,若$f<0$,跳转 (2).
                \item [(2)]$f=f+p$,输出$f$即为$F\mod p$.
            \end{enumerate}

        \subsubsection{特殊情况 ($p=2^{k}, 2\leq k \leq 15$)}
            \[
                F[15:0] \mod p 
                = F[k-1:0]
                = F[15:0] \& (p-1)
            \]
    
    \subsection{数列递推}
    求$F_{N+1}$时,不妨记$F0$存储$F_{N-1}$,$F1$存储$F_{N}$,取$T0 = F0$,$T1 = F1$,
    利用上文实验原理,使得
    \[
        T0 = T0 \mod p,\ 
        T1 = T1 \mod q
    \]
    令$F0 = F1 = F_{N}$,再令$F1 = T0 + T1 = F_{N+1}$,继续执行即可得$F_{N+2}$等.
       
    \subsection{求最大重复字符串长度}
    记Ml为当前重复字符串长度 (初始值为1),Re为最大重复字符串长度 (初始值为1),N为字符串总共长度,i为已判断的字符串数组最大下标
        \subsubsection{特殊情况$(N=0)$}
            若$N=0$,则令$Re = 0$ ,返回$Re$即为结果.

        \subsubsection{更新当前最大重复字符串长度}
            若$Ml > Re$,则令$Re = Ml$;否则不操作.
    
        \subsubsection{判断当前重复字符串长度($N\neq 0$)}
            \begin{enumerate}
                \item [(0)]若$i+1 \geq N$,更新当前最大重复字符串长度,$Ml=1$,跳转 (3);否则跳转 (1)
                \item [(1)]若$s[i] == s[i+1]$,则$Ml++$,跳转 (2) ; 否则更新当前最大重复字符串长度,$Ml=1$,跳转 (2)
                \item [(2)]$i++$,跳转 (0)
                \item [(3)]输出Re即为最终结果,结束程序
            \end{enumerate}

    \subsection{冒泡排序 (升序)}
    对长度为16的整型数组score[\ ]进行升序的冒泡排序
        \begin{lstlisting}[language={[ANSI]C},numbers=left, numberstyle=\tiny, keywordstyle=\color{blue!70}, commentstyle=\color{red!50!green!50!blue!50}, frame=shadowbox, rulesepcolor=\color{red!20!green!20!blue!20}]
            for( int16_t i = 0 ; i < 16 ; i++ ){
                for( int16_t j = 0 ; j < 16 - i ; j++ ){
                    if( score[j] > score[j+1] ){
                        int16_t tem = score[j];
                        score[j] = score[j+1];
                        score[j+1] = tem;
                    }
                }
            }
        \end{lstlisting}
    \subsection{成绩分类}
    用A、B分别记录成绩分类为A、B的人数。
    对前50\% 的成绩进行判断即可,若大于等于85且排在前四名 (25\%),则$A++$,否则$B++$;
    若大于等于75,则$B++$.
    \begin{lstlisting}[language={[ANSI]C},numbers=left, numberstyle=\tiny, keywordstyle=\color{blue!70}, commentstyle=\color{red!50!green!50!blue!50}, frame=shadowbox, rulesepcolor=\color{red!20!green!20!blue!20}]
        for( int16_t i = 15 ; i > 7 ; i-- ){
            if( score[i] >= 85 ){
                if( i > 11 )
                    A++;
                else
                    B++;
            }
            else if ( score[i] >= 75 ){
                B++;
            }
        }
    \end{lstlisting}
    

\section{实验过程}
\subsection{特殊情况的判断}
\begin{enumerate}
    \item [(1)]对于lab3,首先要对$n=0$进行判断,其次在循环到末尾时,要对最后一串重复字符串的长度是否需要更新进行判断
    \item [(2)]lab4中score的排序是在原数组上进行的
\end{enumerate}



\clearpage
\section{代码}    
\begin{lstlisting}[language={[ANSI]C},numbers=left, numberstyle=\tiny, keywordstyle=\color{blue!70}, commentstyle=\color{red!50!green!50!blue!50}, frame=shadowbox, rulesepcolor=\color{red!20!green!20!blue!20}]
    #include <cstdint>
    #include <iostream>
    #include <fstream>
    #define MAXLEN 100
    #ifndef LENGTH
    #define LENGTH 3
    #endif
    
    int16_t lab1(int16_t a, int16_t b) {
        //initialize
        int16_t A =a;
        int16_t B =b;
        int16_t Result =0;
        int16_t I =1;
        //calculation
        for(int16_t i =0 ; i-B <0 ; i++ ){
            if( (I & A) != 0 )
                Result ++;
            I = I + I;
        }
        //return value
        return Result;
    }
    int16_t lab2(int16_t p, int16_t q, int16_t n) {
        //initialize
        int16_t Result = 0;
        int16_t P = p - 1;
        int16_t Q = q;
        int16_t F0 = 0;     //F0 = F_{N-1}
        int16_t F1 = 1;     //F1 = F_{N}
        int16_t N = n;
        //calculation
        while (N!=0){
            int16_t T0 = F0;
            int16_t T1 = F1;
            T0 = T0 & P;
            for( ; T1 >=0 ; ){
                T1 = T1 - Q;
            }
            T1 = T1 + Q;
            F0 = F1;
            F1 = T0 + T1;
            N--;
        }
        Result = F1;
        //return value
        return Result;
    }
    int16_t lab3(int16_t n, char s[]) {
        //initialize
        int16_t Result = 0;
        int16_t N = n;
        int16_t Ml = 1;
        int16_t i = 0;
        char S0,S1;
    
        //calculation
        while( N != 0 ){
            S0 = s[i];
            S1 = s[i+1];
            if( S0 == S1 ){
                Ml ++;
            }
            else{
                if( Ml > Result)
                    Result = Ml;
                Ml = 1;
            }
            i++;
            N--;
        }
        if( Ml > Result )
            Result = Ml;
        //return value
        return Result;
    }
    void lab4(int16_t score[], int16_t *a, int16_t *b) {
        //initialize
        int16_t A = 0;
        int16_t B = 0;
        //calculation
        for( int16_t i = 0 ; i < 16 ; i++ ){
            for( int16_t j = 0 ; j < 16 - i ; j++ ){
                if( score[j] > score[j+1] ){
                    int16_t tem = score[j];
                    score[j] = score[j+1];
                    score[j+1] = tem;
                }
            }
        }
        for( int16_t i = 15 ; i > 7 ; i-- ){
            if( score[i] >= 85 ){
                if( i > 11 )
                    A++;
                else
                    B++;
            }
            else if ( score[i] >= 75 ){
                B++;
            }
        }
        //return value
        * a = A;
        * b = B;
    }
    int main() {
        std :: fstream file;
        file.open("test.txt", std :: ios :: in);
        //lab1
        int16_t a = 0, b = 0;
        for (int i = 0; i < LENGTH; ++i) {
            file >> a >> b;
            std :: cout << lab1(a, b) << std :: endl;
        }
    
        //lab2
        int16_t p = 0, q = 0, n = 0;
        for (int i = 0; i < LENGTH; ++i) {
            file >> p >> q >> n;
            std :: cout << lab2(p, q, n) << std :: endl;
        }
    
        //lab3
        char s[MAXLEN];
        for (int i = 0; i < LENGTH; ++i) {
            file >> n >> s;
            std :: cout << lab3(n, s) << std :: endl;
        }
    
        //lab4
        int16_t score[16];
        for (int i = 0; i < LENGTH; ++i) {
            for (int j = 0; j < 16; ++j) {
                file >> score[j];
            }
            lab4(score, &a, &b);
            for (int j = 0; j < 16; ++j) {
                std :: cout << score[j] << " ";
            }
            std :: cout << std :: endl << a << " " << b << std :: endl;
        }
        file.close();
        return 0;
    }
\end{lstlisting}

\clearpage
\section{实验结果}
\begin{figure*}[htbp]
    \centering
    \includegraphics*[]{r.png}
\end{figure*}

\end{document}
\begin{lstlisting}[language={[ANSI]C},numbers=left, numberstyle=\tiny, keywordstyle=\color{blue!70}, commentstyle=\color{red!50!green!50!blue!50}, frame=shadowbox, rulesepcolor=\color{red!20!green!20!blue!20}]
   
\end{lstlisting}